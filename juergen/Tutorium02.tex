%% LaTeX-Beamer template for KIT design
%% by Erik Burger, Christian Hammer
%% title picture by Klaus Krogmann
%%
%% version 2.0
%%
%% mostly compatible to KIT corporate design v2.0
%% http://intranet.kit.edu/gestaltungsrichtlinien.php
%%
%% Problems, bugs and comments to
%% burger@kit.edu

\documentclass[18pt]{beamer}
\usetheme{kit}

%% TITLE PICTURE

% if a custom picture is to be used on the title page, copy it into the 'logos'
% directory, in the line below, replace 'mypicture' with the 
% filename (without extension) and uncomment the following line
% (picture proportions: 63 : 20, *.eps format if you use latex+dvips+ps2pdf,
% *.jpg/*.png/*.pdf if you use pdflatex)

%\titleimage{mypicture}

%% TITLE LOGO

% for a custom logo on the front page, copy your file into the 'logos'
% directory, insert the filename in the line below and uncomment it

%\titlelogo{mylogo}

% (*.eps format if you use latex+dvips+ps2pdf,
% *.jpg/*.png/*.pdf if you use pdflatex)

%% BIBTEX ICON/KEY

% if you want to see BibTeX keys in the references view instead of the symbol,
% uncomment the following line
% \usebibitemtemplate{\insertbiblabel}

% the presentation starts here

% change the following line to "ngerman" for German style date and logos
% change the following line to "english" for English style date and logos
\selectlanguage{ngerman}

\beamertemplatenavigationsymbolsempty

\usepackage{listings}
\definecolor{darkgray}{rgb}{0.95,0.95,0.95}
\definecolor{darkgreen}{rgb}{0.05,0.7,0.05}
\lstset{ language=Java,
	backgroundcolor=\color{darkgray}, 
	numbers=none, 
	keywordstyle=\color{black}\bfseries,
	tabsize=2,
	showspaces=false,               % show spaces adding particular underscores
	showstringspaces=false,         % underline spaces within strings
	showtabs=false, 
}



\title[Tutorium01]{Tutorium 02: UML in Aktion}
\subtitle{Softwaretechnik im SS 2011, Tutorien 4 + 11 + 17}
\author{Jürgen Walter}
\date{\today}

\institute{Chair for Software Design and Quality}

\begin{document}

%title page
\begin{frame}
\titlepage
\end{frame}

%table of contents
\frame{
\frametitle{Was machen wir heute?}
\tableofcontents
}

\section{Altes Übungsblatt}

\subsection{Altes Übungsblatt}
\frame {
\frametitle{Altes Übungsblatt}
\begin{block}{Aufgabe 1: Mailingliste}
\begin{itemize}
\item ...
\end{itemize}
\end{block}


\begin{block}{Aufgabe 2: Lastenheft}
\begin{enumerate}
\item Zielbestimmung
\item Produkteinsatz
\item Funktionale Anforderungen
\item Produktdaten
\item Nichtfunktionale Anforderungen
\item Systemmodelle
\begin{itemize}
	\item Szenarien
	\item Anwendungsfälle
\end{itemize}
\item Glossar (Begriffslexikon zur Beschreibung des Produktes)
\end{enumerate}
\end{block}
}

\frame {
\begin{block}{Aufgabe 3 Durchführbarkeitsuntersuchung} 
\begin{itemize}
\item jeden der 6 Aspekte ansprechen
\item “Probleme durch ... treten nicht auf, \textbf{da ...}” ist auch eine gute Antwort
\end{itemize}

\end{block}
\begin{block}{Aufgabe 4 Hans Olo}
\begin{itemize}
\item \textbf{Benutzt Checkstyle!}
\end{itemize}
\end{block}

\begin{block}{Aufgabe 5 Vorbereitung der Programmieraufgabe}
\begin{itemize}
\item hat jeder das Projekt runter geladen?
\end{itemize}
\end{block}
}

\subsection{Zum Aufwärmen ...}
\frame {
\frametitle{Wahr oder falsch?}
\begin{itemize}
	\color<2->[rgb]{0,1,0}
	\item In der Planungsphase wird die softwaretechnische Realisierbarkeit eines Produktes untersucht
	\color[rgb]{0,0,0}
	\pause
	\color<3->[rgb]{1,0,0}
	\item Ein Pflichtenheft beschreibt die Eigenschaften, die das Produkt aus der Sicht des Kunden erfüllen soll
	\color[rgb]{0,0,0}
	\pause
	\color<4->[rgb]{0,1,0}
	\item In einem UML-Anwendungsfalldiagramm werden typische Interaktionen des Benutzers mit dem System modelliert
	\color[rgb]{0,0,0}
	\pause
	\color<5->[rgb]{1,0,0}
	\item Das Lastenheft ist eine Verfeinerung des Pflichtenheftes
	\color[rgb]{0,0,0}
	\pause
	\color<6->[rgb]{1,0,0}
	\item Im Pflichtenheft steht beschrieben, wie etwas zu implementieren ist. Es werden z.B. Algorithmen und Datenstrukturen beschrieben
	\color[rgb]{0,0,0}
\end{itemize}
}

\subsection{Pflichtenheft}

\frame{
\frametitle{Pflichtenheft}
\begin{itemize}
\item Das Pflichtenheft ist eine Verfeinerung des Lastenheftes\pause
\item Das Pflichtenheft beschreibt nicht, wie, sondern nur \textbf{was} zu implementieren ist. \pause
\\ $\Rightarrow$ es werden weder Algorithmen noch Datenstrukturen festgelegt \pause
\item das Pflichtenheft definiert das Projekt so vollständig und exakt, \\
dass Entwickler das System implementieren können, \\
ohne nachfragen oder raten zu müssen, \textbf{was} zu implementieren ist.
\end{itemize}
}


\section{Werkzeuge}
\subsection{Versionsverwaltungen}

\frame{
\frametitle{Versionsverwaltungen}

\begin{block}{Subversion}
\begin{itemize}
\item von der Vorlesung unterstützte Versionsverwaltung
\item Windows: Tortoise SVN
\item Linux: Shell oder RabbitVCS
\item Mac: Shell
\end{itemize}
\end{block}
}

\frame {
\frametitle{Klausur 2009}
\begin{block}{Aufgabe}
Erklären Sie die beiden Begriffe „Striktes Ausbuchen“ und „Optimistisches Ausbuchen“ im Kontext einer Konfigurationsverwaltung. Nennen Sie jeweils einen Vor- sowie einen Nachteil. (4P)
\end{block}

Striktes Ausbuchen
\visible <2-> {
\begin{itemize}
\item Nur eine Ausbuchung gleichzeitig ist erlaubt 
\item Ausbucher hat exklusives Änderungsrecht 
\item Vorteil: kein Verschmelzungsaufwand beim Zurückschreiben 
\item Nachteil: immer nur einer kann eine Version ändern
\end{itemize}
}
Optimistisches Ausbuchen 
\visible<3-> {
\begin{itemize}
\item Mehrere Ausbuchungen gleichzeitig erlaubt 
\item Mehrere Entwickler Arbeiten an der gleichen Programmversion 
\item Vorteil: Mehrere Entwickler können eine Version ändern 
\item Nachteil: Aufwand beim Zusammenführen der Versionen (der Schnellere gewinnt)
\end{itemize}
}
}

\subsection{Checkstyle}

\frame{
\frametitle{Checkstyle}
\begin{alertblock}{Das geht besser \dots}
\begin{itemize}
\item Fast keiner von euch ist ohne Checkstyle Fehler
\end{itemize}
\end{alertblock}

\pause
\begin{block}{private Konstruktoren und utility classes}
\begin{itemize}
\item Fehlermeldung: \\“utility classes should not have a public or default constructor” \pause
\item ist alles in einer Klasse \texttt{static}, dann sollte man diese Klasse vermutlich nicht instanziieren \pause
\item durch einen privaten Konstruktor verhindert man das versehentliche Instanziieren:\\
\texttt{ private MyClass() \{\} } \pause
\item eine solche “utility class” sollte als \texttt{final} markiert werden: \\
\texttt{ public final class MyClass \{ }
\end{itemize}
\end{block}
}

\frame{
\frametitle{Checkstyle - }
\begin{block}{weitere Fehler}
\begin{itemize}
\item Fehlermeldung: \\“X sollte auf derselben/ der nächsten Zeile stehen” \pause
\\ in Eclipse mit Autoformat beheben: \texttt{Ctrl + Shift + F} \pause
\item Fehlermeldung: “Javadoc unvollständig” \pause
\\ vor jeder Klasse und jeder Methode ohne Kommentar \texttt{/**} schreiben und  \texttt{Enter} drücken \pause
\item Fehlermeldung: “Importe einzeln angeben” \pause
\\ in Eclipse Importe vervollständigen: \texttt{ Ctrl + Shift + O }
\end{itemize}
\end{block}
}

\subsection{Eclipse}
\frame{
\frametitle{Eclipse Tipps }
\begin{block}{einfaches Umbenennen}
\begin{itemize}
\item es gibt eine einfache Möglichkeit, ein beliebiges Element eines Java Programms in Eclipse umzubenennen \pause
\item dazu markiert man das gewünschte Element und drückt \texttt{ Alt + Shift + R } bzw. wählt “Refactor”$>$”Rename” \pause
\end{itemize}
\end{block}

\begin{block}{SVN in Eclipse}
\begin{itemize}
\item sobald in Eclipse ein SVN Plugin installiert ist, könnt ihr darauf über “Rechtsklick, Team” zugreifen \pause
\item dort findet ihr “Commit”, “Update”, “Show History” und mehr
\end{itemize}
\end{block}

\begin{block}{Local History}
\begin{itemize}
\item Eclipse merkt sich alle Änderungen an Dateien, unabhängig von SVN \pause
\item gelöschte Datien findet man z.B. über “Rechtsklick auf Projekt”, “Restore from Local History”
\end{itemize}
\end{block}
}

\frame{
\frametitle{Checkstyle in Eclipse}

\begin{block}{Ergebnisse anzeigen}
\begin{itemize}
\item sobald Checkstyle richtig konfiguriert wurde, gibt es einige hilfreiche Übersichten
\item \texttt{Window > Show View > Other > Checkstyle}
\item hier gibt es \texttt{Checkstyle violations}, \texttt{Checkstyle violations chart} und \texttt{Duplication Code}
\end{itemize}
\end{block}
}

\frame{
\frametitle{Eclipse - mehr Fehler anzeigen}
\begin{center}
\includegraphics[width=1\textwidth]{pics/checkstyle1}
\end{center}
}
\frame{
\frametitle{Eclipse - mehr Fehler anzeigen}
\begin{center}
\includegraphics[width=0.5\textwidth]{pics/checkstyle2}
\end{center}
}
\frame{
\frametitle{Eclipse - mehr Fehler anzeigen}
\begin{center}
\includegraphics[width=0.5\textwidth]{pics/checkstyle3}
\end{center}
}

\section{UML}

\subsection{Aktivitätsdiagramm}

\frame{
\frametitle {Aktivitätsdiagramm} 
\begin{itemize}
	\item Ein Aktivitätsdiagramm beschreibt einen Ablauf
	\item Besteht aus \textbf{Aktions-, Objekt-} und \textbf{Kontrollflussknoten} sowie \textbf{Objekt-} und 											\textbf{Kontrollflüssen}
	\item Elemente eines Aktivitätsdiagramms
	\begin{itemize}
		\item Aktionen
		\item Knoten (Startknoten, Endknoten, Ablaufende)
		\item Entscheidungen (durch Raute dargestellt)
		\item Zusammenführung
		\item Aufteilung des Kontrollflusses
		\item Synchronisation
	\end{itemize}
\end{itemize}
}

\frame {
\begin{exampleblock}{Beispiel Aktivitätsdiagramm}
\begin{center}
\includegraphics[scale=0.22]{pics/beispiel.png}
\end{center}
\end{exampleblock}
}

\frame{
\begin{center}
\includegraphics[scale=0.5]{pics/diagramm.png}
\end{center}
}

\section{Ende}

\subsection{Tipps zum nächsten Übungsblatt}

\frame{
\frametitle{Tipps zum nächsten Übungsblatt}

\begin{block}{Aufgabe 1 - Klassendiagramm}
\begin{itemize}
\item denkt an Attribute, Multiplizitäten, Restriktionen, Assoziationsnamen sowie Rollen \pause
\item merkt euch den Unterschied zwischen Komposition und Aggregation!
\end{itemize}
\end{block}

\begin{block}{Aufgabe 2 - Aktivitätsdiagramm}
\begin{itemize} \pause
\item ihr dürft die Aufgabe auf zwei Diagramme verteilen \pause
\item \url{http://de.wikipedia.org/wiki/Floyd-Steinberg-Algorithmus}
\end{itemize}
\end{block}

}


\frame{
\frametitle{Tipps zum nächsten Übungsblatt}
\begin{block}{Aufgabe 3 - Programmieren}
\begin{itemize}
\item args4j kann euch sehr viel Arbeit ersparen, versucht die Bedienung anhand von Configuration.java zu verstehen 
\item achtet darauf ob ihr Ganzzahldivision verwendet: 
\\ \pause
\texttt{int o; \dots; int p = (o + 128 ) / 256 * 255 \\
		// liefert hier 0 oder 255: wieso?}
\end{itemize}
\end{block}
}


\frame{
\frametitle{Bis zum nächsten Mal}
\begin{center}
\includegraphics[width=1\textwidth]{pics/time_management}
\end{center}

}


\end{document}
