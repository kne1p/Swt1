%% LaTeX-Beamer template for KIT design
%% by Erik Burger, Christian Hammer
%% title picture by Klaus Krogmann
%%
%% version 2.0
%%
%% mostly compatible to KIT corporate design v2.0
%% http://intranet.kit.edu/gestaltungsrichtlinien.php
%%
%% Problems, bugs and comments to
%% burger@kit.edu

\documentclass[18pt]{beamer}
\usetheme{kit}

%% TITLE PICTURE

% if a custom picture is to be used on the title page, copy it into the 'logos'
% directory, in the line below, replace 'mypicture' with the 
% filename (without extension) and uncomment the following line
% (picture proportions: 63 : 20, *.eps format if you use latex+dvips+ps2pdf,
% *.jpg/*.png/*.pdf if you use pdflatex)

%\titleimage{mypicture}

%% TITLE LOGO

% for a custom logo on the front page, copy your file into the 'logos'
% directory, insert the filename in the line below and uncomment it

%\titlelogo{mylogo}

% (*.eps format if you use latex+dvips+ps2pdf,
% *.jpg/*.png/*.pdf if you use pdflatex)

%% BIBTEX ICON/KEY

% if you want to see BibTeX keys in the references view instead of the symbol,
% uncomment the following line
% \usebibitemtemplate{\insertbiblabel}

% the presentation starts here

% change the following line to "ngerman" for German style date and logos
% change the following line to "english" for English style date and logos
\selectlanguage{ngerman}

\beamertemplatenavigationsymbolsempty

\usepackage{listings}
\definecolor{darkgray}{rgb}{0.95,0.95,0.95}
\definecolor{darkgreen}{rgb}{0.05,0.7,0.05}
\lstset{ language=Java,
	backgroundcolor=\color{darkgray}, 
	numbers=none, 
	keywordstyle=\color{black}\bfseries,
	tabsize=2,
	showspaces=false,               % show spaces adding particular underscores
	showstringspaces=false,         % underline spaces within strings
	showtabs=false, 
}



\title[Tutorium04]{Tutorium 04: Entwurfsmuster}
\subtitle{Softwaretechnik im SS 2011, Tutorium 4}
\author{Jürgen Walter}
\date{\today}

\institute{Chair for Software Design and Quality}

\begin{document}

%title page
\begin{frame}
\titlepage
\end{frame}

%table of contents
\frame{
\frametitle{Was machen wir heute?}
	\tableofcontents
}

\section{Altes Übungsblatt}

\subsection{Altes Übungsblatt}
\frame {
\frametitle{Altes Übungsblatt}
	\begin{block}{Aufgabe 1 - UML Zustandautomaten}
	\begin{itemize}
	\item ???
	\end{itemize}
	\end{block}
}


\begin{frame}[fragile]
\frametitle{Altes Übungsblatt}
	\begin{block}{Aufgabe 2 - Überladen und Überschreiben}
	\begin{itemize}
	\item ???
	\end{itemize}
	\end{block}
	\pause
	\begin{block}{Aufgabe 3 - JMJRST erweitern} 
	\begin{itemize}
	\item ???
	\end{itemize}
	\end{block}
\end{frame}

\frame{
\frametitle{Altes Übungsblatt}
	\begin{block}{Bonusaufgabe - Kammerjäger}
	\begin{itemize}
	\item wer hat den Debugger benutzt?
	\item bla?
	\end{itemize}
	\end{block}
}

\subsection{Zum Aufwärmen ...}
\frame {
\frametitle{Wahr oder falsch?}
\begin{itemize}
	\color<2->[rgb]{1,0,0}
	\item Fassade und Adapter gehören zu den Bequemlichkeitsmustern.
	\color[rgb]{0,0,0}
	\pause
	\color<3->[rgb]{0,1,0}
	\item Das Entwurfsmuster „Abstrakte Fabrik“ konzentriert sich auf den schrittweisen Konstruktionsprozess komplexer Objekte. geändert werden.
	\color[rgb]{0,0,0}
	
	\pause
	\color<4->[rgb]{0,1,0}
	\item Eine Fabrikmethode kann eine Einschubmethode bei einer Schablonenmethode für Objekterzeugung sein.
	\color[rgb]{0,0,0}
	\pause
	\color<5->[rgb]{1,0,0}
	\item Software ist leichter zu ändern als ein physisches Produkt vergleichbarer Komplexität.
	\color[rgb]{0,0,0}
	\pause
	\color<6->[rgb]{1,0,0}
	\item In Java ist das Entwurfsmuster „Null-Objekt“ durch das Schüsselwort null realisiert.
	\color[rgb]{0,0,0}

\pause
	\color<7->[rgb]{0,1,0}
	\item Unter welchen Umständen ein Objekt welche Botschaft entgegen nimmt, spezifiziert man in einem UML-Zustandsdiagramm
	\color[rgb]{0,0,0}

\pause
	\color<8->[rgb]{0,1,0}
	\item Zusicherungen (z.B. mit dem Schlüsselwort assert in Java) werden zur Laufzeit eines Programs ausgeführt.
	\color[rgb]{0,0,0}
\end{itemize}
}

\frame {
\frametitle {Klausuraufgaben zum Aufwärmen} 
	\begin{block} {Aufgabe 1 (1P)}
	Nennen Sie genau zwei Gründe, die laut Vorlesung für den Einsatz von Entwurfsmustern sprechen. \\
	\visible<2-> {
	\begin{itemize}
		\item verbessern die Kommunikation im Team
		\item erfassen wesentliche Konzepte und bringen sie in eine verständliche Form
		\item helfen Entwürfe zu verstehen
		\item dokumentieren Entwürfe kurz und knapp
		\item dokumentieren und fördern den Stand der Kunst
		\item helfen weniger erfahrenen Entwerfern)
		\item vermeiden die Neuerfindung des Rades
		\item können Code-Qualität und Code-Struktur verbessern
		\item fördern gute Entwürfe und guten Code durch Angabe konstruktiver Beispiele
	\end{itemize}
	}
	\end{block}
}

\frame {
\frametitle {Klausuraufgaben zum Aufwärmen} 
	\begin{block} {Aufgabe 2 (2P)}
	In der Konfigurationsverwaltung haben Sie zwei Alternativen kennengelernt, Änderungen an versionierten Dateien zu speichern. Nennen und beschreiben Sie die beiden Alternativen kurz. Nennen Sie für jede Alternative je einen Vorteil.\\
	\visible<2-> {
	\begin{itemize}
		\item Vorwärtsdeltas speichern eine Grundversion und die daran durchgeführten Änderungen
		\item Rückwärtsdeltas speichern die aktuelle Version und die Änderungen für frühere Versionen
		\item Vorteil Vorwärtsdelta: Schneller Zugriff auf frühere Versionen
		\item Vorteil Rückwärtsdelta: Schneller Zugriff auf aktuelle Versionen
	\end{itemize}
	}
	\end{block}
}

\frame {
\frametitle {Klausuraufgaben zum Aufwärmen} 
	\begin{block} {Aufgabe 3 (3P)}
In der Vorlesung wurden für den Vergleich zweier Objekte mehrere Stufen von Gleichheit definiert. Geben Sie die Definitionen für Gleichheit 0. und 1. Stufe an. \\
	\visible<2-> {
	\begin{itemize}
		\item Gleichheit 0. Stufe: Es handelt sich um dasselbe Objekt, die Objekte sind identisch.
		\visible<3-> {
		\item Gleichheit 1. Stufe: Es handelt sich entweder um dasselbe Objekt oder zwei verschiedene Objekte, die aber in allen Attributen/Assoziationen identische Werte besitzen (Gleichheit 0. Stufe oder paarweise Gleichheit 0. Stufe in allen Attributen)
		}
	\end{itemize}
	}
	\end{block} 
}

\section{Polymorphie}
\frame {
\frametitle {Polymorphie} 
	\begin{block} {Polymorphie}
	\begin{itemize}
		\item Polymorphie bedeutet Vielgestaltigkeit
	\end{itemize}
	\end{block} 

	\begin{block} {dynamische Polymorphie}
	\begin{itemize}
		\item Verwendung von Vererbung
	\end{itemize}
	\end{block} 

	\begin{block} {statisch Polymorphie}
	\begin{itemize}
		\item Überladen von Methoden
	\end{itemize}
	\end{block} 
}


\subsection{Vererbung}


\frame {
\frametitle {Vererbung} 
	\begin{block} {Signaturvererbung}
	\begin{itemize}
		\item Oberklasse gibt an, welche Methoden implementiert werden sollen
		\item Java: Implementieren eines Interfaces mit implements
	\end{itemize}
	\end{block} 

	\begin{block} {Implementierungsvererbung}
	\begin{itemize}
		\item Oberklasse hat Methoden schon implementiert
		\item Methode kann überschrieben werden (@Override)
		\item Java: Erweitern einer Klasse mit extends
	\end{itemize}
	\end{block} 

	\begin{block} {Frage}
	\begin{itemize}
		\item Gibt es eine Einschränkungen für die Vererbung?
		\visible<2-> {
		\item Ja, das Substitutionsprinzip
		}
	\end{itemize}
	\end{block} 
}

\subsection{Überladen}
\frame {
\frametitle{Überladen}
	\begin{block}{Methoden überladen \dots}
	\begin{itemize}
	\item statische Polymorpie
	\item Methoden einer Klasse mit gleichen Namen, aber unterschiedlicher Signatur
	\item Hat nichts mit Vererbung zu tun!
	\item Bequem für den Benutzer
	\end{itemize}
	\end{block}
}


\section{UML}
\subsection{Zustandsdiagramm}

\frame{
\frametitle {Zustandsdiagramm (Wiederholung)} 
	\begin{itemize}
	\item Beschreibt mögliche Zustände eines Objekts sowie mögliche Zustandsübergänge
	\item Der Zustandsübergang (Transition) wird durch ein Ereignis ausgelöst
	\begin{center}
	\includegraphics[scale=0.5]{pics/zustandsuebergang.png}
	\end{center}
	\item Zustandsübergang findet nur statt, wenn die Bedingung zu diesem Zeitpunkt erfüllt ist
	\item $\epsilon$ -Übergänge sind erlaubt
	\end{itemize}
}

\frame {
	\begin{exampleblock}{Zustandsdiagramm mit Gedächtnis (Wiederholung)}
	\begin{center}
	\includegraphics[scale=0.6]{pics/ZustandsDiagrammGedaechtnis.png}
	\end{center}
	\end{exampleblock}
}


\frame{
\frametitle {Klausuraufgabe 2009 (1P)}
	Gegeben ist der folgende UML-Zustandsautomat. Geben Sie an, in welcher Zustandskombination	
	sich der Zustandsautomat, jeweils ausgehend vom Startzustand, nach den beiden Eingabefolgen
	befindet. 
	\begin{itemize}
	\item Folge1: a, b, c, c   
	\visible<2-> {
	\item Zustandskombination: $A \times $D
	}
	\item Folge2: c, c, a, b, b, a, c, c, a
	\visible<3-> {
	\item Zustandskombination: $B \times $C
	}
	\end{itemize}
	\begin{center}
	\includegraphics[scale=0.6]{pics/03/zustandN.png}
	\end{center}
}

\section{Ende}

\subsection{Tipps zum nächsten Übungsblatt}

\frame{
\frametitle{Tipps zum nächsten Übungsblatt}

	\begin{block}{Aufgabe 1 - Entwurfsmuster in der Java-API}
	\begin{itemize}
	\item Falls ihr nicht wisst, was die jeweiligen Code-Stellen tun, "googelt" euch Anwendungsbeispiele
	\item Stellt wirklich nur das allernotwendigste mit UML dar, denn diese Klassen sind recht groß
	\end{itemize}
	\end{block}

	\begin{block}{Aufgabe 2 - Kreuzworträtsel}
	\begin{itemize} \pause
	\item das sollte jeder mit den Entwurfsmustern aus der Vorlesung hinbekommen
	\item achtet darauf, den jeweiligen Namen aus der Vorlesung zu benutzen!
	\end{itemize}
	\end{block}
}


\frame{
\frametitle{Tipps zum nächsten Übungsblatt}
	\begin{block}{Aufgabe 3 - Entwurfsmuster anwenden}
	\begin{itemize}
	\item in beiden Projekten soll genau ein passendes Entwurfsmuster umgesetzt werden, um das Programm einfacher bzw. einfacher erweiterbar zu machen
	\item Abgabe besteht jeweils aus zwei Teilen:

		\begin{itemize}
		\item je ein Klassendiagramm auf Papier
		\item je ein Zip-Archiv mit dem umgebauten Programmcode
		\end{itemize}

	\end{itemize}
	\end{block}
}


\frame{
\frametitle{Bis zum nächsten Mal}
	\begin{center}
	\includegraphics[height=200pt]{pics/04/04_comic}
	\end{center}
}

\end{document}
