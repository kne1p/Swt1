%% LaTeX-Beamer template for KIT design
%% by Erik Burger, Christian Hammer
%% title picture by Klaus Krogmann
%%
%% version 2.0
%%
%% mostly compatible to KIT corporate design v2.0
%% http://intranet.kit.edu/gestaltungsrichtlinien.php
%%
%% Problems, bugs and comments to
%% burger@kit.edu

\documentclass[18pt]{beamer}
\usetheme{kit}

%% TITLE PICTURE

% if a custom picture is to be used on the title page, copy it into the 'logos'
% directory, in the line below, replace 'mypicture' with the 
% filename (without extension) and uncomment the following line
% (picture proportions: 63 : 20, *.eps format if you use latex+dvips+ps2pdf,
% *.jpg/*.png/*.pdf if you use pdflatex)

%\titleimage{mypicture}

%% TITLE LOGO

% for a custom logo on the front page, copy your file into the 'logos'
% directory, insert the filename in the line below and uncomment it

%\titlelogo{mylogo}

% (*.eps format if you use latex+dvips+ps2pdf,
% *.jpg/*.png/*.pdf if you use pdflatex)

%% BIBTEX ICON/KEY

% if you want to see BibTeX keys in the references view instead of the symbol,
% uncomment the following line
% \usebibitemtemplate{\insertbiblabel}

% the presentation starts here

\title[Short title]{Tutorium 01: Projektplanung }
\subtitle{Softwaretechnik im SS 2011, Gruppe 2}
\author{Jürgen Walter}

\institute{Chair for Software Design and Quality}

\begin{document}

% change the following line to "ngerman" for German style date and logos
% change the following line to "english" for English style date and logos
\selectlanguage{ngerman}

%title page
\begin{frame}
\titlepage
\end{frame}

%table of contents
\frame{
\frametitle{Was machen wir heute?}
\tableofcontents
}



\section{Organisatorisches}

\subsection{Vorstellung}
\frame{
\frametitle{Wer bin ich?}
\begin{itemize}
\item Jürgen Walter
\pause
\item 8tes Semester Informatik
\pause
\item juergen.walter.halle@gmail.com
\item uxccx@student.kit.edu
\pause
\item Partnerturoren Christian Juelg und Daniel Deckers
\item \dots
\end{itemize}
}

\subsection{Übungsschein}
\frame{
\frametitle{Übungsschein}
\begin{block}{Der Übungschein ist \dots}
\begin{itemize}
\pause
\item Vorraussetzung zur Klausur
\pause
\item 6 Übungsblätter
\pause
\item 150 Punkte insgesammt 
\pause
\item mit 50 Prozent aus Übungsblättern und Programmmieraufgaben bestanden 
\end{itemize}
\end{block}
}

\section{Zum Aufwärmen ...}
\frame {
\frametitle{wahr falsch}
\begin{itemize}
	\item In der Planungsphase wird die softwaretechnische Realisierbarkeit eines Produktes untersucht
	\pause
	\item Ein Pflichtenheft beschreibt die Eigenschaften, die das Produkt aus der Sicht des Kunden erfüllen soll
	\pause
	\item In einem UML-Anwendungsfalldiagramm werden typische Interaktionen des Benutzers mit dem System modelliert
\end{itemize}
}

\section{Versionsverwaltungen}

\frame{
\frametitle{Subversion}
\begin{itemize}
\pause
\item von der Vorlesung unterstützte Versionsverwaltung
\end{itemize}
}

\frame{
\frametitle{Git}
\begin{itemize}
\pause
\item Werkzeug des Tutors ;-)
\end{itemize}
}



\section{Lastenheft}

\subsection{Gliederungsschema}

\frame{
\frametitle{Gliederungsschema}
\begin{itemize}
\item enthält die Hauptanforderungen an das Produkt (user requirements)
\item formuliert mit natürlicher Sprache, evtl. Diagramme
\item dient der Kommunikation mit dem Kunden und der Projektplanung
\end{itemize}
}

\frame{
\frametitle{Gliederungsschema}
\begin{block}{Gliederungsschema Lastenheft}
\begin{itemize}
\item1.Zielbestimmung
\item2.Produkteinsatz
\item3.Produktfunktionen
\item4.Produktdaten
\item5.Produktleistungen
\item6.Qualitätsanforderungen
\item7.Ergänzungen (weitere nichtfunktionale Eigenschaften).
\item8.Glossar (Begriffslexikon zur Beschreibung des Produktes)
\end{itemize}
\end{block}
}



\section{Ende}

\frame{
\frametitle{Nächstes Übungsblatt}
}


\frame{
\frametitle{Ende}
\begin{center}
\includegraphics[width=0.4\textwidth]{pics/chuckNorris}
\end{center}

}


\end{document}
