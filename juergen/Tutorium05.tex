%% LaTeX-Beamer template for KIT design
%% by Erik Burger, Christian Hammer
%% title picture by Klaus Krogmann
%%
%% version 2.0
%%
%% mostly compatible to KIT corporate design v2.0
%% http://intranet.kit.edu/gestaltungsrichtlinien.php
%%
%% Problems, bugs and comments to
%% burger@kit.edu

\documentclass[18pt]{beamer}
\usetheme{kit}

%% TITLE PICTURE

% if a custom picture is to be used on the title page, copy it into the 'logos'
% directory, in the line below, replace 'mypicture' with the 
% filename (without extension) and uncomment the following line
% (picture proportions: 63 : 20, *.eps format if you use latex+dvips+ps2pdf,
% *.jpg/*.png/*.pdf if you use pdflatex)

%\titleimage{mypicture}

%% TITLE LOGO

% for a custom logo on the front page, copy your file into the 'logos'
% directory, insert the filename in the line below and uncomment it

%\titlelogo{mylogo}

% (*.eps format if you use latex+dvips+ps2pdf,
% *.jpg/*.png/*.pdf if you use pdflatex)

%% BIBTEX ICON/KEY

% if you want to see BibTeX keys in the references view instead of the symbol,
% uncomment the following line
% \usebibitemtemplate{\insertbiblabel}

% the presentation starts here

% change the following line to "ngerman" for German style date and logos
% change the following line to "english" for English style date and logos
\selectlanguage{ngerman}

\beamertemplatenavigationsymbolsempty

\usepackage{listings}
\definecolor{darkgray}{rgb}{0.95,0.95,0.95}
\definecolor{darkgreen}{rgb}{0.05,0.7,0.05}
\lstset{ language=Java,
	backgroundcolor=\color{darkgray}, 
	numbers=none, 
	keywordstyle=\color{black}\bfseries,
	tabsize=2,
	showspaces=false,               % show spaces adding particular underscores
	showstringspaces=false,         % underline spaces within strings
	showtabs=false, 
}



\title[Tutorium05]{Tutorium 05: Parallelismus und Testen}
\subtitle{Softwaretechnik im SS 2011, Tutorium 4}
\author{Jürgen Walter}
\date{\today}

\institute{Chair for Software Design and Quality}

\begin{document}

%title page
\begin{frame}
\titlepage
\end{frame}

%table of contents
\frame{
\frametitle{Was machen wir heute?}
	\tableofcontents
}

\section{Altes Übungsblatt}

\subsection{Altes Übungsblatt}
\frame {
\frametitle{Altes Übungsblatt}
	\begin{block}{Aufgabe 1 - Entwurfsmuster in der Java-API}
	\begin{itemize}
	\item ???
	\end{itemize}
	\end{block}

	\begin{block}{Aufgabe 2 - Kreuzworträtsel}
	\begin{itemize} \pause
	\item geschenkte Punkte \pause
	\item der Zusammenhang zwischen der Beschreibung und dem Muster sollte euch dennoch klar werden!
	\end{itemize}
	\end{block}
}


\begin{frame}[fragile]
\frametitle{Altes Übungsblatt}
	\begin{block}{Aufgabe 3 - Entwurfsmuster anwenden}
	\begin{itemize}
	\item ???
	\end{itemize}
	\end{block}
\end{frame}


\subsection{Zum Aufwärmen ...}
\frame {
\frametitle{Wahr oder falsch?}
\begin{itemize}
	\color<2->[rgb]{1,0,0}
	\item Die letzte Phase des klassischen Wasserfallmodells ist „Testen und Abnahme“.
	\color[rgb]{0,0,0}
	\pause
	\color<3->[rgb]{0,1,0}
	\item Regressionstests helfen verhindern, dass alte Fehler wieder auftreten.
	\color[rgb]{0,0,0}
	
	\pause
	\color<4->[rgb]{0,1,0}
	\item Eine Fabrikmethode kann eine Einschubmethode bei einer Schablonenmethode für Objekterzeugung sein.
	\color[rgb]{0,0,0}
	\pause
	\color<5->[rgb]{1,0,0}
	\item In Java muss eine Klasse, die eine Schnittstelle implementiert, alle in der Schnittstelle vorgegebenen Methoden implementieren
	\color[rgb]{0,0,0}
	\pause
	\color<6->[rgb]{1,0,0}
	\item In Java ist das Entwurfsmuster „Null-Objekt“ durch das Schüsselwort null realisiert.
	\color[rgb]{0,0,0}

\pause
	\color<7->[rgb]{0,1,0}
	\item Wenn eine Klasse eine abstrakte Methode besitzt, dann ist sie auch selbst abstrakt.
	\color[rgb]{0,0,0}

\pause
	\color<8->[rgb]{0,1,0}
	\item Zusicherungen (z.B. mit dem Schlüsselwort assert in Java) werden zur Laufzeit eines Programs ausgeführt und überprüft.
	\color[rgb]{0,0,0}
\end{itemize}
}

\frame {
\frametitle {Klausuraufgaben zum Aufwärmen} 
	\begin{block} {Aufgabe 1 (1P)}
Was ist der Nachteil von Zyklen in der Benutzt-Relation zwischen Modulen? \\
	\visible<2-> {
	Die einzelnen Module können nicht nacheinander implementiert
und getestet werden, weil ihr Funktionieren von einer korrekten
Implementierung aller Module des Zyklus abhängt. Oder:
„Nothing works until everything works.“
	}
	\end{block}
}

\frame {
\frametitle {Klausuraufgaben zum Aufwärmen} 
	\begin{block} {Aufgabe 2 (2P)}
Nennen Sie jeweils zwei in der Vorlesung besprochene Entwurfsmuster der Kategorien Entkopplungsmuster und Variantenmuster und ordnen Sie die genannten Entwurfsmuster der entsprechenden Kategorie zu. \\

	
	\begin{itemize}
		\item Entkopplungsmuster: 
		\visible<2-> {Adapter, Beobachter, Brücke, Iterator, Stellvertreter, Vermittler}
		\item Variantenmuster:
		\visible<3-> {Abstrakte Fabrik, Besucher, Erbauer, Fabrikmethode, Kompositum, Schablonenmethode, Strategie, Dekorierer}
	\end{itemize}
	
	\end{block}
}

\begin{frame}[fragile]
\frametitle {Klausuraufgaben zum Aufwärmen} 
	\begin{block} {Aufgabe 3 (1P)}

	\begin{lstlisting} {}
	public static double blub(double[] d) {
		if  (d != null \&\& d.length > 0) {
		...
		}
		...
	}
	\end{lstlisting}
	
	Begründen Sie: Was wäre die Folge, wenn man das \&\& durch ein \& ersetzt?
	\visible<2-> {
	Keine Kurzauswertung (0,5 P) 
	$\Rightarrow$ Bei null als Eingabe gäbe es eine NullPointerExeption bei d.length (0,5 P).
	}
	\end{block} 
\end{frame}



\frame {
\frametitle{Aufgabe}
\section{Kontrollflussorientiertes Testen}

Gegeben sei der folgende Algorithmus, der Werte aus dem RGB-Farbraum in Werte des HSV-Farbraums umrechnet.
\begin{center}
\includegraphics[scale=0.45]{pics/05/code.png}
\end{center}


}

\frame {
\frametitle{Aufgabe a)}

a) Übersetzen sie den Algorithmus in die in der Vorlesung besprochene Zwischensprache und erstellen sie daraus den Kontrollflussgraphen
\begin{center}
\includegraphics[scale=0.45]{pics/05/code.png}
\end{center}

}

\frame {
\frametitle {Musterlösung Teil a)}
\begin{center}
\includegraphics[scale=0.45]{pics/05/testM.png}
\end{center}
}
\frame {
\frametitle{Aufgabe Teil b )+ c))}
Für gültige Eingabewerte gilt: $r, g, b \in  [0, 1]$; für gültige Ausgaben: $s, v \in [0, 1]$ und $h \in [0, 360]$.

b) Geben Sie eine minimale Testfallmenge an, welche die Anweisungsüberdeckung erfüllt und eine, welche die Zweigüberdeckung erfüllt. 
Geben Sie für jeden Testfall den durchlaufenen Pfad im Kontrollflussgraphen an. \\
Anweisungsüberdeckung:
\visible<2-> {
 $\{1.0, 0.0, 0.5\} \rightarrow 1, 2, 3, 7, 8, 9$ $ \{0.0, 1.0, 0.0\}\rightarrow 1, 2, 4, 5, 7, 9 $ $\{0.0, 0.0, 1.0\} \rightarrow 1, 2, 4, 6, 7, 9 $
 } \\

Zweigüberdeckung:
 \visible<3-> {
siehe oben + $\{0.0, 0.0, 0.0\} \rightarrow 1, 9$ \\
}
c) Welches Problem gibt es bei der Pfadüberdeckung für diesen Algorithmus?
\visible<4-> {
Pfadüberdeckung: einige Pfade sind bei korrekten Eingaben nicht ausführbar. Wenn max $\neq$r ist, kann $h<0$ nicht wahr werden
}


}
\section{Ende}
\subsection{Tipps zum nächsten Übungsblatt}

\frame{
\frametitle{Tipps zum nächsten Übungsblatt}

	\begin{block}{Aufgabe 1 - Kontrollflussorientiertes Testen}
	\begin{itemize}
	\item kommt oft in der Klausur vor \pause
	\item relativ leicht verdiente Punkte!
	\end{itemize}
	\end{block}
}

\frame{
\frametitle{Tipps zum nächsten Übungsblatt}

	\begin{block}{Aufgabe 2 -Gebietszerlegung und Synchronisation}
	\begin{itemize} \pause
	\item a) sollte jeder hinbekommen - falls euch nichts einfällt wählt einfach ein paar Parameter und schaut was passiert \pause
	\item b) sieht auf den ersten Blick harmlos aus - um sicher zu gehen programmiert ein Testprogramm und testet, ob euer Fix tatsächlich das Problem behebt \pause
	\item stellt dazu auch sicher, dass die ursprüngliche Barriere tatsächlich falsch funktioniert in eurem Test-Programm \pause
	\item ein Test ist nur dann sinnvoll, wenn er den Unterschied zwischen richtigem und falschem Code auch zeigt - \pause
	\item sonst kann es sein, dass euer Test das eigentliche Fehlverhalten nicht sichtbar macht
	\end{itemize}
	\end{block}
}


\frame{
\frametitle{Tipps zum nächsten Übungsblatt}
	\begin{block}{Aufgabe 3 - Parallelisierung}
	\begin{itemize}
	\item für diese Aufgabe gibt es 9 ``Pflicht''- und 5 Bonuspunkte
	\item ihr braucht zum Testen einen Rechner mit mindestens zwei Kernen - \\
		falls ihr keinen habt, gibt es in der Atis genug davon (Core2Duo, Core i5, Athlon X2)
	\item zuverlässige Laufzeitmessungen kann man oft brauchen, und es gibt dabei mehr als genug Fehlerquellen ;-)
	\end{itemize}
	\end{block}
}


\frame{
\frametitle{Bis zum nächsten Mal}
	\begin{center}
	\includegraphics[height=200pt]{pics/05/05_comic}
	\end{center}
}

\end{document}
