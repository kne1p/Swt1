%% LaTeX-Beamer template for KIT design
%% by Erik Burger, Christian Hammer
%% title picture by Klaus Krogmann
%%
%% version 2.0
%%
%% mostly compatible to KIT corporate design v2.0
%% http://intranet.kit.edu/gestaltungsrichtlinien.php
%%
%% Problems, bugs and comments to
%% burger@kit.edu

\documentclass[18pt]{beamer}
\usetheme{kit}

%% TITLE PICTURE

% if a custom picture is to be used on the title page, copy it into the 'logos'
% directory, in the line below, replace 'mypicture' with the 
% filename (without extension) and uncomment the following line
% (picture proportions: 63 : 20, *.eps format if you use latex+dvips+ps2pdf,
% *.jpg/*.png/*.pdf if you use pdflatex)

%\titleimage{mypicture}

%% TITLE LOGO

% for a custom logo on the front page, copy your file into the 'logos'
% directory, insert the filename in the line below and uncomment it

%\titlelogo{mylogo}

% (*.eps format if you use latex+dvips+ps2pdf,
% *.jpg/*.png/*.pdf if you use pdflatex)

%% BIBTEX ICON/KEY

% if you want to see BibTeX keys in the references view instead of the symbol,
% uncomment the following line
% \usebibitemtemplate{\insertbiblabel}

% the presentation starts here

% change the following line to "ngerman" for German style date and logos
% change the following line to "english" for English style date and logos
\selectlanguage{ngerman}

\beamertemplatenavigationsymbolsempty

\usepackage{listings}
\definecolor{darkgray}{rgb}{0.95,0.95,0.95}
\definecolor{darkgreen}{rgb}{0.05,0.7,0.05}
\lstset{ language=Java,
	backgroundcolor=\color{darkgray}, 
	numbers=none, 
	keywordstyle=\color{black}\bfseries,
	tabsize=2,
	showspaces=false,               % show spaces adding particular underscores
	showstringspaces=false,         % underline spaces within strings
	showtabs=false, 
}



\title[Tutorium06]{Tutorium 06: Klausurvorbereitung}
\subtitle{Softwaretechnik im SS 2011, Tutorium 4}
\author{Jürgen Walter}
\date{\today}

\institute{Chair for Software Design and Quality}

\begin{document}

%title page
\begin{frame}
\titlepage
\end{frame}

%table of contents
\frame{
\frametitle{Was machen wir heute?}
	\tableofcontents
}

\section{Rückblick}

\frame{
\frametitle{Rückblick ÜB6}

	\begin{block}{Aufgabe 1 - Kontrollflussorientiertes Testen}
	\begin{itemize}
	\item kommt oft in der Klausur vor \pause
	\item relativ leicht verdiente Punkte!
	\end{itemize}
	\end{block}
}

\frame{
\frametitle{Rückblick Blatt 6}

	\begin{block}{Aufgabe 2 -Gebietszerlegung und Synchronisation}
	\begin{itemize} \pause
	\item b) falls eine ähnliche Aufgabe in der Klausur vorkommt - versucht schnell eine Lösung zu finden \pause \\
		wenn ihr nicht schnell eine findet, macht besser andere Aufgaben zuerst \pause
	\item wichtig: wait() kann auch ``grundlos'' enden, daher sollte wait() immer in einer Schleife stehen
	\end{itemize}
	\end{block}
}


\frame{
\frametitle{Rückblick Blatt 6}
	\begin{block}{Aufgabe 3 - Parallelisierung}
	\begin{itemize}
	\item Auch Parallelisierung kann in der Klausur vorkommen \pause
	\item merkt euch insbesondere die parallelen Entwurfsmuster und die wichtigsten Java Konstrukte
	\end{itemize}
	\end{block}
}

\frame {
\frametitle{Wahr oder falsch?}
\begin{itemize}
	\color<2->[rgb]{0,1,0}
	\item Software unterliegt keinem Verschleiß.
	\color[rgb]{0,0,0}
	\pause
	\color<3->[rgb]{0,1,0}
	\item „Synchronisiere und Stabilisiere“ erzielt die Effektivität durch kurze Entwicklungsphasen.
	\color[rgb]{0,0,0}
	
	\pause
	\color<4->[rgb]{1,0,0}
	\item Kontrollflussorientierte Tests gehören zu den statischen Testverfahren.
	\color[rgb]{0,0,0}
	\pause
	\color<5->[rgb]{0,1,0}
	\item Bei der testgetriebenen Entwicklung dienen die Tests unter anderem zur Schnittstellendefinition
	\color[rgb]{0,0,0}
	\pause
	\color<6->[rgb]{1,0,0}
	\item Im Prozessmodell „Prototyp“ wird der Prototyp iterativ entwickelt und nach der Testphase produktiv eingesetzt und gewartet
	\color[rgb]{0,0,0}
	\pause
	\color<7->[rgb]{1,0,0}
	\item Die Signatur einer Methode besteht aus dem Methodennamen und dem Rückgabetyp
	\color[rgb]{0,0,0}


\end{itemize}
}

\section{Klausurvorbereitung}
\subsection{Parallelisierung}
\frame {
\frametitle {Matrix-Matrix-Multiplikation} 
	\begin{block} {Varianten}
	Es gibt viele Möglichkeiten, um $C = A x B$ zu berechnen
	\begin{itemize}
		\item klassisch bzw. ijk-Algorithmus
		\item ikj- oder kij-Reihenfolge
		\item Systolisch
		\item Cannon-Algorithmus
	\end{itemize}
	\end{block}
}

\frame {
\frametitle {Matrix-Matrix-Multiplikation} 
	\begin{block} {klassisch}
	\begin{center}
	\includegraphics[width=\textwidth]{pics/06/matrix_klassisch}
	\end{center}
	Problem: cache-unfreundlich
	\end{block}
}

\begin{frame}[fragile]
\frametitle {Matrix-Matrix-Multiplikation} 
	\begin{block} {Quellcode zu ``ijk''}

	\begin{lstlisting} {}
public int[][] matrixMult (int[][] a, int[][] b) {
  final int N = a.length;
  int[][] c = new int[N][N];
  
  for(int i = 0; i < N; i++) {
    for(int j = 0; j < N; j++) {
      for(int k = 0; k < N; k++) {
        c[i][j] += a[i][k] * b[k][j];
      }
    }
  }
  return c;
}
	\end{lstlisting}
	\end{block} 
\end{frame}

\begin{frame}[fragile]
\frametitle {Matrix-Matrix-Multiplikation} 
	\begin{block} {Quellcode zu ``ikj''}
durch Neuordnung der Schleifen wird ein Operand schleifeninvariant \pause
	\begin{lstlisting} {}
for(i=0; i < N; i++)
  for(k=0; k < N; k++)
    for(j=0; j < N; j++)
      c[i][j] += a[i][k] * b[k][j];
	\end{lstlisting}
der invariante Teil muss nur einmal berechnet werden \pause
	\begin{lstlisting} {}
for(i=0; i < N; i++)
  for(k=0; k < N; k++) {
    r = a[i][k];
    for(j=0; j < N; j++)
      c[i][j] += r* b[k][j];
  }
	\end{lstlisting}
	\end{block} 
\end{frame}

\frame {
\frametitle {Matrix-Matrix-Multiplikation} 
	\begin{center}
	\includegraphics[width=0.6\textwidth]{pics/06/matrix_ikj}
	\end{center}
	\begin{block} {ikj-Algorithmus}
	\begin{itemize}
		\item cache-freundlich \pause
		\item weniger parallelisierbar als ijk - weshalb?
		\visible<3-> {
		\item es können nur noch die $N$ Iterationen der äußeren Schleife parallel ohne Synchronisation für C 			bearbeitet werden
		}
	\end{itemize}
	\end{block}
}


\subsection{Singleton}
\begin{frame}[fragile]
\frametitle {Klausur 2010-1} 
	\begin{block} {A3 b)}
	Ergänzen Sie den folgenden Quelltext der Klasse \texttt{Singleton}, so dass sie das Entwurfsmuster ``Einzelstück'' implementiert. Zudem soll Ihre Implementierung für die mehrfädige Ausführung geeignet sein. Verwenden Sie korrekte Java-Syntax. (3 P)
	\end{block}
	
	\begin{lstlisting} {}
/**
 * Implementiert ein fuer mehrfaedige Programme 
 * geeignetes Einzelstueck
 */
public class Singleton {



}
	\end{lstlisting}
\end{frame}

\begin{frame}[fragile]
\frametitle {Klausur 2010-1 A3 b) Lösung} 
	
	\begin{lstlisting} {}
public class Singleton {
  // Instanzvariable Einzelstuecks
  private static Singleton instanz = null;
  
  // Default-Konstruktor privat deklarieren
  private Singleton() {}
  
  // Statische Methode, die die Instanz zurueckgibt
  public static synchronized Singleton getInstance() {
    if (instanz == null) {
      instanz = new Singleton();
  }
  return instanz;
}
	\end{lstlisting}
\end{frame}


\subsection{Prozessmodelle}
\frame {
\frametitle {Prozessmodelle} 
	\begin{block} {Aufgabe 1 (2,5P)}
	Nennen Sie 5 aus der Vorlesung bekannte Prozessmodelle zur Softwareentwicklung.\\
	\begin{itemize}
		\visible<2-> {
		\item Programmieren durch Probieren
		\item Wasserfallmodell
		\item V-Modell
		\item Prototypenmodell
		\item Iteratives Modell
		\item Synchronisiere und Stabilisiere
		\item Agile Methoden (spez. Extreme Programming)
		}
	\end{itemize}
	\end{block}
}

\frame {
\frametitle {Prozessmodelle} 
	\begin{block} {Programmieren durch Probieren (Trail and Error)}
		pro
		\begin{itemize}
			\item einfach
			\item schnell, ohne "nutzlose" Dokumentation
		\end{itemize}
		contra
		\begin{itemize}
			\item mangelnde Aufgabenerfüllung, da keine Anforderungsanalyse
			\item kostspielige Wartung
			\item Teamarbeit nicht vorgesehen
		\end{itemize}
	\end{block}
}

\frame {
\frametitle {Prozessmodelle} 
	\begin{block} {Wasserfallmodell}
	\begin{itemize}
		\item sequentiell durch alle Phasen durch
	\end{itemize}

	pro
	\begin{itemize}
		\item einfach
	\end{itemize}

	contra
	\begin{itemize}
		\item keine Phasenübergreifende Rückkopplung
		\item schwer Parallelisierbar
		\item zwingt Benutzer alles am Anfang zu verstehen
	\end{itemize}

	\end{block}
}

\frame {
\frametitle {Prozessmodelle} 
	\begin{block} {V-Modell XT}
		Allgemein
		\begin{itemize}
			\item keine Reihenfolge bzw keine Phasengrenzen
			\item definiert stattdessen Rollen
		\end{itemize}
		Produktzustände
		\begin{itemize}
			\item in Bearbeitung
			\item vorgelegt
			\item fertig gestellt
		\end{itemize}
		\begin{center}	
			\includegraphics[height=70pt]{pics/06/VModellXT}
		\end{center}
	\end{block}
}

\frame {
\frametitle {Prozessmodelle} 
	\begin{block} {V-Modell}
		\begin{itemize}
			\item Streng genommen KEIN Prozessmodell, sondern Zuordnung von Artefakten
		\end{itemize}
		\begin{center}	
			\includegraphics[height=150pt]{pics/06/VModell}
		\end{center}
	\end{block}
}

\frame {
\frametitle {Prozessmodelle} 
	\begin{block} {Prototypen-Modell}
		WICHTIG
		\begin{itemize}
			\item Prototypen Wegwerfen
		\end{itemize}
		pro
		\begin{itemize}
			\item Test of Concept
			\item stärkt Arbeitsmoral
			\item stärkt Vertrauen zwischen Kunden und Entwickler
		\end{itemize}
	\end{block}

	\begin{block} {Iteratives Modell}
		\begin{itemize}
			\item Erweiterung der Prototypenidee, ohne Wegwerfen
			\item gleiche Vor und Nachteile wie bei Prototypen
			\item Definierte Planung und Analysephase (im Vergleich zu Trail and Error)
		\end{itemize}
	\end{block}

}

\frame {
\frametitle {Prozessmodelle} 
	\begin{block} {Synchronisiere und Stabilisiere (Microsoft)}
		Ansatz
		\begin{itemize}
			\item organisiere die vielen Entwickler in kleine Teams - Freiheit für eigene Ideen/Entwürfe
			\item Aber: Synchronisiere regelmäßig (nächtlich)
			\item Und Stabilisiere regelmäßig (Meilensteine, 3 Mon.)
		\end{itemize}

		3 Phasen
		\begin{itemize}
			\item Planungsphase
			\item Entwicklungsphase
			\item Stabilisierungsphase
		\end{itemize}
	\end{block}
}
		

\frame {
\frametitle {Prozessmodelle} 
	\begin{block} {Synchronisiere und Stabilisiere (Microsoft) II}
		pro
		\begin{itemize}
			\item kurze Produkzyklen
			\item Priorisierung nach Funktionen
			\item Natürliche Modularisierung nach Funktionen
			\item Fortschritt auch ohne vollständige Spezifikation möglich
			\item Viele Entwickler arbeiten in kleinen Teams und damit genau so
			effektiv wie wenige
			\item Rückmeldungen können frühzeitig einfließen
		\end{itemize}
	\end{block}
}
		


\frame {
\frametitle {Prozessmodelle} 
	\begin{block} {Synchronisiere und Stabilisiere (Microsoft) III}
		contra
		\begin{itemize}
			\item Ungeeignet für manche Art von Software – Architekturprobleme,
			mangelhafte Fehlertoleranz, Echtzeitfähigkeit
			\item Mündliche Arbeitsweise: ad-hoc-Prozesse in jedem Team, kein
			Lernen über Teamgrenzen
			\item Alle 18 Monate sind 50 Prozent des Codes überarbeitet worden

		\end{itemize}
	\end{block}
}



\subsection{Kontrollflussorientierte Testverfahren}
\frame {
\frametitle{Kontrollflussorientierte Testverfahren}%Klausur 2009-1 A2 a) 
	\begin{center}
	\includegraphics[width=\textwidth]{pics/06/kfg_a2a}
	\end{center}
}

\frame {
\frametitle {Klausur 2009-1 A2 a) Lösung} 
	\begin{center}
	\includegraphics[width=\textwidth]{pics/06/kfg_a2a_lsg}
	\end{center}
}

\frame {
\frametitle {Bedingungs-Überdeckungen} 
	\begin{center}
	\includegraphics[width=0.9\textwidth]{pics/06/kfg_hierarchie}
	\end{center}
}

\frame {
\frametitle {Klausur 2009-1 A2 b)} 
	\begin{center}
	\includegraphics[width=\textwidth]{pics/06/kfg_a2b}
	\end{center}
}

\frame {
\frametitle {Klausur 2009-1 A2 b) Lösung} 
	\begin{center}
	\includegraphics[width=\textwidth]{pics/06/kfg_a2b_lsg}
	\end{center}
}

\frame {
\frametitle {Aufwandsabschätzung} 
	\begin{block} {Aufgabe 2 (2P)}
		Nennen Sie 2 aus der Vorlesung bekannte Basismethoden der Aufwandsschätzung und erklären Sie eine davon kurz. \\
		\begin{itemize}
		\visible<2-> {
			\item Analogiemethode
			\item Relationsmethode
			\item Multiplikatormethode
			\item Phasenaufteilung
			}
		\end{itemize}
	\end{block}
}

\frame {
\frametitle {Testen} 
	\begin{block} {Aufgabe 2 (3P)}
		Nennen Sie die 3 aus der Vorlesung bekannten Arten von Testhelfern und erklären Sie diese jeweils in einem Satz. \\
		\begin{itemize}
		\visible<2-> {
			\item Stummel (stub)  \visible<3-> {Rudimentär implementierte Methode als Teil der Software. Dient als Platzhalter für 	noch nicht umgesetzte Funktionalität.}
			\item Attrappe (dummy):  \visible<4->{Simuliert die Implementierung zu Testzwecken}
			\item Nachahmung (mock):  \visible<5->{Attrappe mit zusätzlicher Funktionalität, wie bspw. das Einstellen der Reaktion der Nachahmung auf bestimmte Eingaben oder das Überprüfen des Verhaltens des „Klienten“.}
			}
		\end{itemize}
	\end{block}
}

\section{inoffizielle Probeklausur}

\frame{
\frametitle{inoffizielle Probeklausur}

	\begin{block}{Ablauf}
	\begin{itemize}
	\item Zeit: 25.7. 18 Uhr \pause
	\item Ort: Audimax
	\item die Tutoren korrigieren die Probeklausuren \pause
	\item Rückgabe in der gleichen Woche, vorraussichtlich Freitag \pause
	\item die Probeklausur wird von Tutoren ausgerichtet, es gibt keinen Zusammenhang mit der offizellen Klausur!
	\end{itemize}
	\end{block}
\pause
	\begin{block}{Anmeldung}
	\begin{itemize}
	\item damit wir genügend Kopien drucken können, meldet euch bitte jetzt an \pause
	\item damit die Probeklausur euch hilft, müsst ihr schon vorher mit der Klausurvorbereitung anfangen
	\end{itemize}
	\end{block}
}


\frame{
\frametitle{Bis zum nächsten Mal}
	\begin{center}
	\includegraphics[height=200pt]{pics/06/06_comic}
	\end{center}
}

\end{document}
