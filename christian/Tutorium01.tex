\input{static/start.tex}

\title[Short title]{Tutorium 01: Projektplanung }
\subtitle{Softwaretechnik im SS 2011, Tut 17}
\author{Christian Jülg}
\date{28.04.2011}

\institute{Chair for Software Design and Quality}

\begin{document}

%title page
\begin{frame}
\titlepage
\end{frame}

%table of contents
\frame{
\frametitle{Was machen wir heute?}
\tableofcontents
}



\section{Organisatorisches}

\subsection{Vorstellung}
\frame{
\frametitle{Wer bin ich?}
\begin{itemize}
\item Christian Jülg
\pause
\item 14-tes Semester Informatik
\pause
\item swt-tutor@gmx.de
\pause
\item Partnerturoren Jürgen Walter und Daniel Deckers
\item \dots
\end{itemize}
}

\subsection{Übungsschein}
\frame{
\frametitle{Übungsschein}
\begin{block}{Der Übungschein ist \dots}
\begin{itemize}
\pause
\item Vorraussetzung zur Klausur
\pause
\item 6 Übungsblätter
\pause
\item 150 Punkte insgesamt 
\pause
\item mit 50 Prozent aus Übungsblättern und Programmmieraufgaben bestanden 
\end{itemize}
\end{block}
}



\section{Versionsverwaltungen}

\frame{
\frametitle{Subversion}
\begin{itemize}
\pause
\item von der Vorlesung unterstützte Versionsverwaltung
\end{itemize}
}

\frame{
\frametitle{Git}
\begin{itemize}
\pause
\item Werkzeug des Tutors ;-)
\end{itemize}
}



\section{Lastenheft}

\subsection{Gliederungsschema}

\frame{
\frametitle{Gliederungsschema}
\begin{itemize}
\item enthält die Hauptanforderungen an das Produkt (user requirements)
\item formuliert mit natürlicher Sprache, evtl. Diagramme
\item dient der Kommunikation mit dem Kunden und der Projektplanung
\end{itemize}
}

\frame{
\frametitle{Gliederungsschema}
\begin{block}{Gliederungsschema Lastenheft}
\begin{itemize}
\item1.Zielbestimmung
\item2.Produkteinsatz
\item3.Produktfunktionen
\item4.Produktdaten
\item5.Produktleistungen
\item6.Qualitätsanforderungen
\item7.Ergänzungen (weitere nichtfunktionale Eigenschaften).
\item8.Glossar (Begriffslexikon zur Beschreibung des Produktes)
\end{itemize}
\end{block}
}



\section{Ende}

\frame{
\frametitle{Nächstes Übungsblatt}
}


\frame{
\frametitle{Ende}
\begin{center}
\includegraphics[width=0.4\textwidth]{pics/chuckNorris}
\end{center}

}


\end{document}
